%!TEX root = main.tex
%----------------------------------------------------------------------------------------
% Configuración
%----------------------------------------------------------------------------------------
\documentclass[11pt, oneside]{book} %Las tesis se imprimen a una cara, por eso está el "oneside"
\usepackage[paperwidth=17cm, paperheight=22.5cm, bottom=2.5cm, right=2.5cm]{geometry} % Configuración de la página
\usepackage{amssymb, amsmath, amsthm} %Paquetes para símbolos matemáticos
\usepackage[spanish, mexico]{babel} %Para que el texto esté en español
\usepackage[utf8]{inputenc} %Para acentos y símbolos
\usepackage{enumerate, enumitem} %Para enumerar páginas
\usepackage{graphicx, pgfplots, pgfplotstable, filecontents, tikz, listings, color, float} %Para gráficos y figuras
\usepackage{csquotes, multirow, libertine, adjustbox, threeparttable} % Para citas y tablas
\usepackage[nottoc]{tocbibind}
\usepackage{mdframed} % marco para teoremas y lemas
% \usepackage[backend=bibtex, style=authoryear, autocite=inline]{biblatex} % Para otro tipo de citación
%\usepackage[backend=biber,style=ieee,citestyle=authoryear]{biblatex} % Para citación utilizando APA
%\DeclareLanguageMapping{spanish}{spanish-ieee}
\pgfplotsset{compat=newest}

% Para hipervínculos
 \usepackage{hyperref}
 \hypersetup{
     colorlinks=false,
     linkcolor=black,
     pdftitle={Tesis},
     pdfauthor={Alan J. Amaya Martínez},
     %bookmarks=true,
     linktocpage=true,
     urlcolor=black,
     allbordercolors = {white}
}

\renewcommand{\lstlistingname}{Algoritmo}

\addto\captionsspanish{
    \renewcommand{\contentsname}
    {Tabla de contenido}
}

\pgfplotsset{compat=newest}

%\addbibresource{FormatoTesis/Referencias/references.bib} %Inserción del archivo de bibliografía
% \nocite{*}

\graphicspath{{Imagenes/}} %Carpeta donde están las imágenes

% Para eliminar guiones y justificar texto
\tolerance=1
\emergencystretch=\maxdimen
\hyphenpenalty=10000
\hbadness=10000
\linespread{1.25} %Para tener interlineado de 1.5 de Word

% Define el estilo del teorema
\theoremstyle{definition}
\newtheorem{theorem}{Teorema}[section]
\newtheorem{lemma}[theorem]{Lema}

% Envuelve en marco
\surroundwithmdframed[
  linewidth=0.5pt,
  linecolor=black,
  topline=true,
  bottomline=true,
  leftline=true,
  rightline=true,
  innertopmargin=8pt,
  innerbottommargin=8pt,
  innerleftmargin=8pt,
  innerrightmargin=8pt
]{theorem}

\surroundwithmdframed[
  linewidth=0.5pt,
  linecolor=black,
  topline=true,
  bottomline=true,
  leftline=true,
  rightline=true,
  innertopmargin=8pt,
  innerbottommargin=8pt,
  innerleftmargin=8pt,
  innerrightmargin=8pt
]{lemma}

% ---- Prueba en lema/teorema ----
\makeatletter
\renewenvironment{proof}[1][\proofname]{\par
  \pushQED{\qed}%
  \normalfont \topsep6\p@\@plus6\p@\relax
  \trivlist
  \item[\hskip\labelsep
        \textbf{#1:}]\ignorespaces  % Cambiado a \textbf
}{%
  \popQED\endtrivlist\@endpefalse
}
\makeatother


%---- INICIA EL DOCUMENTO -------
\begin{document}

%----------------------------------------------------------------------------------------
% Portada
%----------------------------------------------------------------------------------------
\begin{titlepage}
\begin{center}
\textsc{\Large Instituto Tecnológico Autónomo de México}\\[1em]

%Logo ITAM
\begin{figure}[h]
\begin{center}
\includegraphics[scale=0.15]{logo-ITAM.png}
\end{center}
\end{figure}

\textsc{\huge \textbf{Diseño de un Controlador de Velocidad en un Aerogenerador}}\\[2em]

\textsc{\large Tesis}\\[1em]
\textsc{\large que para obtener el título de}\\[1em]
\textsc{\large Ingeniero en Mecatrónica}\\[1em]
\textsc{\large Presenta}\\[1em]
\textsc{\Large Alan Josabet Amaya Martínez}\\[1em]
\textsc{\large Asesor}\\[1em]
\textsc{\large Dr. Rafael Cisneros Montoya}\\[1em]
  
\end{center}
\vspace*{\fill}
\textsc{Ciudad de México \hspace*{\fill} 2024}
\end{titlepage}

\tableofcontents % Tabla de Contenidos
\mainmatter % Empieza la numeración arábiga de las páginas
%----------------------------------------------------------------------------------------
%	Tesis
%----------------------------------------------------------------------------------------
\chapter{Introducción} \label{intro_chapter}

En este capítulo se hace un breve recorrido por la situación actual de la energía eólica en el país. Además, se introducen los conceptos básicos y principios de funcionamiento de la energía eólica y de los aerogeneradores. Se presentan el problema del control de una turbina eólica y la motivación en el diseño de un controlador que permita agregar un grado de libertad al sistema de la turbina. 
\\

\noindent En esta tesis se utilizarán indistintamente los términos aerogenerador y turbina eólica para referirse al conjunto completo del generador eléctrico, las aspas, el cuerpo de la turbina y demás componentes que se detallarán más adelante. 


\section{Contexto}
\noindent En la última década en México se ha visto un incremento en el consumo de energía primaria proveniente de fuentes renovables. Del 2012 al 2022 se registró un incremento de 5.24\% a 8.96\% la proporción del consumo de energía primaria cuyo origen se puede considerar renovable (i.e. energía solar, hidráulica, eólica, geotérmica, bioenergía y energía oceánica). Este incremento es en realidad una recuperación de las energías renovables a proporciones que habían sido observadas con anterioridad. En 1970 cerca del 11\% de la energía primaria consumida en el país provenía de fuentes renovables y desde entonces este indicador se ha mantenido oscilando entre 4-11\% \cite{statistical-review}.
\\

\noindent De forma similar, la producción de energía eléctrica en México proveniente de fuentes renovables a partir de 1985 y hasta 2022 ha fluctuado entre 15-32\% \cite{statistical-review}. En 2022 se registró una participación del 22.94\% de energías renovables en la generación de electricidad \cite{owid-renewable-energy}.
\\

\noindent A nivel mundial, la mayor parte de la generación de electricidad por medio de fuentes renovables se logra por medio de energía hidráulica - 4334.19 TWh (Terawatts hora) en 2022 -, y en segundo lugar se encuentra la energía eólica con 2104.84 TWh \cite{owid-renewable-energy}.
\\

\noindent En México se replica este patrón con 35.72 TWh generados por energía hidráulica y 20.32 TWh a partir de la energía del viento en este mismo año (2022) \cite{owid-renewable-energy}. Así mismo, la participación de la energía eólica en la generación de electricidad en el país pasó de 1.29\% en 2012 a 5.79\% en 2022. Desde el 2010 la generación de electricidad a través de energía eólica ha mostrado un crecimiento promedio de casi 35.49\% anual y este mismo periodo de tiempo solo ha mostrado una reducción en la generación entre el 2021 y el 2022 \cite{statistical-review}.
\\

\noindent De forma similar, la capacidad instalada de las centrales eólicas, definida como “la potencia que tiene una central eléctrica para generar electricidad considerando la disponibilidad técnica de sus instalaciones y de los insumos energéticos que serán transformados en electricidad en dichas instalaciones \cite{CONACYT2022}”, se ha incrementado a una tasa promedio de 28.72\% anual entre el 2010 y el 2022 \cite{statistical-review}.

\section{Breve introducción a las turbinas eólicas}

{\parindent0pt El uso del viento como fuente de energía no es reciente. Al contrario, su uso era conocido en varias de las civilizaciones anteriores a la era cristiana. Durante la edad media y hasta el siglo XVIII fueron populares los molinos de viento en varios países de Europa, principalmente los Países Bajos y Alemania. Este tipo de molinos también son los antecesores directos de las turbinas eólicas de eje horizontal que vemos hoy en día y que se encuentran en prácticamente todas las centrales de producción eléctrica en gran escala alrededor del mundo \cite{Hau2013}.
\\

El uso del viento en la generación de electricidad parece haber iniciado en Estados Unidos con la creación del primer aerogenerador de 12 kW por parte de Charles Bush \cite{Burton2011}, aunque también se consideran a otros inventores, como el danés Poul la Cour, como algunos de los pioneros en utilizar la energía del viento para generar electricidad \cite{Hansen2015}.
\\

Las turbinas eólicas modernas conservan el principio de funcionamiento de los primeros molinos de viento. Esto es, aprovechar la fuerza del viento para generar empuje en las aspas y producir un movimiento rotacional con el cual se pueda generar electricidad. Una mayor velocidad rotacional es deseable dado que reduce la relación de la caja de cambios del rotor. Sin embargo, como se verá más adelante, una velocidad rotacional demasiado alta puede generar demasiado estrés en el generador u otros componentes y conducir a fallos por parte del aerogenerador. 
\\

En 2009 la Asociación Europea de Energía Eólica (EWEA) realizó el balance de energía asociado a un aerogenerador de 3 MW concluyendo que el tiempo promedio en el que dicho aerogenerador produciría la energía equivalente a su producción, operación, transporte y desmantelamiento es de 6 a 7 meses \cite{Burton2011}.
\\

Para la producción de energía a gran escala son más comunes los aerogeneradores de eje horizontal que aquellos de eje vertical (véase figura \ref{VAWT}). Lo anterior debido a distintos factores que favorecen la configuración horizontal del generador entre los que destacan:

\begin{itemize}
\item La posibilidad de controlar la velocidad y potencia otorgadas por el generador a partir de modificar el ángulo de ataque de las aspas del generador, lo cual resulta ser la forma más eficiente de limitar la velocidad angular particularmente en turbinas de gran tamaño.
\item La forma de las aspas se puede diseñar para que sea aerodinámica y, de esta forma, aprovechar mejor toda la energía del viento
\item La existencia de más y mejor tecnología para turbinas de eje horizontal \cite{Hau2013}.
\end{itemize}

\begin{figure}[H]
    \centering
    \includegraphics[width=0.9\textwidth]{Imagenes/Eole_cap-chat.jpeg}
    \caption{\small{\textbf{Aerogeneradores de eje vertical y horizontal:} Pierre5018, CC BY-SA 4.0 (https://creativecommons.org/licenses/by-sa/4.0), vía Wikimedia Commons}}
    \label{VAWT}
\end{figure}

Los aerogeneradores de eje horizontal tienen 3 componentes principales: la torre, el rotor y la góndola (\emph{nacelle}). Dentro de la góndola se encuentra el generador eléctrico conectado con el rotor por medio de los ejes de alta y baja velocidad, así como la caja de cambios (\emph{gearbox}) \cite{Pao2009}. La mayoría de las turbinas eólicas de gran escala cuentan con un sistema que permite girar la góndola y el rotor para que apunte siempre en la dirección del viento, este sistema es manejado por el \emph{yaw actuator} y une la góndola con la torre. El rotor incluye las aspas y, en algunos casos, los actuadores que controlan el ángulo de estas (\emph{pitch actuator}). 
\\

Las turbinas eólicas pueden ser de velocidad fija o variable. Las turbinas de velocidad variable puede trabajar más cerca de la máxima eficiencia aerodinámica por un mayor tiempo. Sin embargo, el hecho de tener una velocidad variable implica también que se debe realizar una correcta regulación y procesamiento de la electricidad generada para que esta pueda ser subida a la red con la frecuencia correcta \cite{Pao2009}.
\\

Las turbinas de velocidad variable suelen trabajar en 3 regiones de operación. Cuando la velocidad del viento es baja (usualmente menor a 6 $\text{m}/\text{s}$) la potencia del viento es insuficiente y las turbinas suelen permanecer detenidas. A esto se le conoce como \textbf{región 1} e incluye también el momento en el que las turbinas se encienden y comienzan a girar. Durante la región 1 el control se limita al monitoreo y supervisión de la velocidad del viento para determinar en qué momento se pueden iniciar las rutinas de activación para encender las turbinas \cite{Johnson2004}.
\\

La \textbf{región 2} de operación ocurre generalmente cuando el viento alcanza velocidades entre 5 y 14 $\text{m}/\text{s}$. En esta región se busca extraer la mayor potencia posible del viento y para esto se pueden utilizar cualquiera de los tipos de control disponibles en el aerogenerador (\emph{yaw control}, \emph{pitch control} y \emph{torque control}). Debido a que las velocidades del viento son menores que las presentadas en la región 3 normalmente no es necesario reducir las cargas [mecánicas y eléctricas] cuando la turbina opera en la región 2 \cite{Johnson2004}.
\\

Por último, la \textbf{región 3} ocurre cuando el viento supera la velocidad a la cual se extrae la potencia máxima por el aerogenerador. Aunque la velocidad del viento siga aumentando, las turbinas deben limitar la potencia que se extrae del viento para evitar daños en sus componentes debido al estrés y las cargas producidas por la fuerza del viento \cite{Pao2009}-\cite{Johnson2004}. De aquí que la gráfica potencia-velocidad se vea plana en la región 3 (ver figura \ref{Regiones}) aun cuando el viento se siga incrementando hasta alcanzar un cierto valor límite (\emph{high wind cutout}). Después de este límite las turbinas deben ser detenidas completamente.
\\
\begin{figure}[H]
    \centering
    \includegraphics[width=0.8\textwidth]{Imagenes/Regions.jpeg}
    \caption{Regiones de control de un aerogenerador \cite{Johnson2004}}
    \label{Regiones}
\end{figure}

Adicionalmente, las turbinas eólicas tienen una eficiencia definida como el porcentaje de la potencia del viento que de hecho puede ser aprovechada por la turbina eólica. La eficiencia máxima alcanzable de manera teórica está dada por el número de Betz $B=\frac{16}{27}\approx 59.26$\% \cite{Huleihil2012}. Esto implica que solo el 59\% de la potencia del viento puede ser realmente convertida en energía eléctrica. Este valor es teórico y en la práctica las turbinas eólicas suelen tener valores de eficiencia entre 35-45\%.
}

\section{Identificación del problema}
{\parindent0pt
El problema del control de una turbina eólica puede ser tratado desde distintos ángulos. Aquella magnitud física que se maneje como variable de control definirá el alcance y el tipo del control que se vaya a realizar. 
\\

Las turbinas eólicas producidas en los últimos años incluyen actuadores para cada una de las aspas, lo que provee una nueva forma de controlar la generación de energía además de la forma tradicional que implica el control del par eléctrico del generador \cite{Laks2009}.
\\

Durante el ciclo de operación de un aerogenerador, este debe poder adaptarse a los cambios constantes e impredecibles del viento, tanto en su dirección como en su velocidad. Para esto se han implementado sistemas de control que van desde el \emph{yaw control} que permite que el conjunto completo de la góndola y el rotor giren sobre la torre para que mantener el viento normal al plano del rotor; hasta el \emph{pitch control} que permite girar las aspas para modificar el ángulo con el que el viento choca contra la superficie aerodinámica de cada aspa. 
\\

Otra forma de enfrentar el problema de la velocidad del viento —y, por tanto, del torque recibido por el rotor— es a través de controlar el par eléctrico del generador (\emph{torque control}), lo que permite controlar la cantidad de torque que se demanda del rotor y de esta forma optimizar su velocidad \cite{Johnson2004}.
\\

Cuando el viento muestra velocidades por debajo del límite establecido para ciertos aerogeneradores en específico, estos suelen realizar el control de velocidad del rotor por medio del \emph{yaw control} y el control del par eléctrico, manteniendo el ángulo de ataque de las aspas a un valor fijo calculado como el óptimo en el cual se extrae la mayor cantidad de energía del viento. Sin embargo, cuando el viento alcanza velocidades más altas que el valor límite se vuelve importante reducir las cargas mecánicas y eléctricas para no superar sus valores máximos considerados en el diseño de dichos componentes. 
\\

La principal consecuencia de alcanzar las velocidades límites del viento es que usualmente conlleva a que los aerogeneradores sean frenados hasta quedar completamente detenidos. Lo anterior para evitar daños a sus componentes debido a las cargas excesivas, considerando además que las turbinas eólicas modernas tienen costos bastante altos asociados tanto a su producción como a su mantenimiento. Esto puede resultar contraproducente dado que los aerogeneradores permanecen detenidos en aquellos momentos en que la generación de energía eléctrica es teóricamente mayor y, por tanto, se está desperdiciando toda esa potencia del viento.


}


\section{Objetivo}
\noindent De lo anterior, la presente tesis tiene como objetivos los siguientes puntos:
\begin{enumerate}
\item Encontrar una ley de control que, manipulando el ángulo de ataque de las aspas, permita regular el par mecánico.
\item Considerar el escenario en el que se requiere extraer la máxima potencia proveniente del viento, procedimiento conocido como MPTT (Maximum Power Point Tracking). Este punto requiere que el aerogenerador rote a cierta velocidad que en la práctica se desconoce.
\item Agregar al modelo la parte eléctrica y encontrar una ley de control que permita regular el par eléctrico al manipular la carga. Luego, hacer operar de manera conjunta ambas partes, la eléctrica y la mecánica, con sus respectivas leyes de control. 
\end{enumerate}


\section{Metodología}

\noindent Para el desarrollo de esta tesis se seguirán la metodología ágil. En particular la variante de la metodología ágil conocida como SCRUM. La elección debido a que su organización en \emph{sprints} cortos con versiones listas para entregar permite enfocar la atención en un problema a la vez y trabajar sobre este hasta resolverlo. De esta forma el esfuerzo se concentra en una etapa del proyecto y hasta que esta esté concluida se avanza con las posteriores. La metodología ágil SCRUM facilita también el desarrollo de proyectos de alta complejidad y en los que se incursiona en áreas poco conocidas para los colaboradores. 
%!TEX root = ../main.tex
\chapter{Análisis}

En este capítulo se introduce formalmente el problema de control del \emph{pitch angle} en las turbinas eólicas. Se detallan las aproximaciones que ha habido para los distintos tipos de control que se pueden realizar en una turbina eólica (principalmente \emph{pitch control} y \emph{torque control}) a través de los años y los resultados que han tenido (\emph{state of the art}). Además, se enuncian los requerimientos y las restricciones para el diseño del controlador. Por último, se mencionan los estándares que existen tanto a nivel nacional como internacional que estén relacionados o que puedan afectar el diseño del controlador.

\section{Requerimientos}
{\parindent0pt
El controlador debe responder adecuadamente ante los cambios en las condiciones del viento. Se debe monitorear (sensar) la velocidad para determinar en qué momento se pasa de una región de operación a otra. El \emph{pitch control} debe responder a la señal del control supervisor cuando la turbina entre en la región 3 de operación y su función es la de disminuir (o aumentar) la potencia recibida del viento. Para esto, cuando el control supervisor determine que la carga mecánica en el rotor es cercana a un valor límite (con una tolerancia para fines de seguridad) el \emph{pitch control} debe abatir las aspas, reduciendo así el ángulo de ataque de estas, y permitiendo que el viento pase por el área de barrido del rotor generando un menor empuje en las aspas. Por el contrario, cuando la velocidad del rotor haya disminuido por debajo de un cierto umbral, las aspas deben girar nuevamente a la posición de máxima extracción de potencia. 
\\

La tarea de control es permitir que el aerogenerador siga operando por un mayor tiempo durante condiciones climáticas con vientos de alta velocidad antes de tener que ser frenado totalmente para evitar daños. Esto tiene como objetivo aprovechar de manera más eficiente los escenarios con vientos más rápidos para extraer en estos momentos mayor energía que durante la operación del aerogenerador en condiciones normales. Dichas condiciones de viento “extremo” suelen ocurrir con baja frecuencia a lo largo de un cierto periodo de tiempo (por lo general un año), y dependen principalmente de la ubicación geográfica del parque eólico donde esté instalada la turbina. Parte fundamental del desarrollo de este trabajo implica el diseño de un controlador que sea factible de implementar en cualquier ubicación geográfica, pero particularmente en aquellas donde se presentan vientos de alta velocidad con mayor frecuencia a lo largo del año.
\\

Se requiere, de igual forma, evaluar la conveniencia de la implementación del sistema de control en términos de su gasto energético en comparación con el incremento en la producción debido a su uso. Esto pensando en que el hecho de agregar un grado de libertad a partir de controlar el \emph{pitch angle} debería provocar un incremento en la generación de energía eléctrica mayor a su consumo debido al uso de los actuadores —consumo que debe ser bajo debido a la naturaleza hidráulica del mecanismo—. Sí lo anterior no se cumple, lo cual puede deberse a la baja frecuencia de vientos de alta velocidad o a la ineficiencia del controlador para aprovechar dichos vientos, entonces se concluiría que el grado extra de libertad debido al control del ángulo de las aspas no es adecuado para un aerogenerador con sus condiciones particulares. 

}

\section{Restricciones}
{\parindent0pt
Dado que el objetivo de la tesis es el diseño de un controlador (a partir de encontrar la ley de control) para el ángulo de ataque de las aspas de la turbina (\emph{pitch angle}) y debido a lo complicado —o incluso inviable— de ser implementado en un aerogenerador real, el resultado esperado para esta tesis será la simulación de la operación de la turbina bajo diferentes condiciones que permita observar el trabajo de los sistemas de control durante las 3 regiones de operación de la turbina. Para esto la simulación debe permitir variar las condiciones del viento simuladas (velocidad y dirección) a lo largo de un periodo determinado de tiempo y, como resultado, se debe observar el comportamiento esperado de la turbina de acuerdo a la región en la que se encuentre operando. 
\\

Es de particular interés la región 3 de operación debido a que es donde el viento tiene velocidades más altas e interesa observar el trabajo del controlador para evaluar su eficiencia al momento de reducir la carga mecánica que recibe el rotor debido al viento. Dado que uno de los objetivos de la tesis es extender el tiempo de operación de la turbina en la región 3 (para incrementar la cantidad de energía eléctrica producida) es indispensable que la simulación muestre de manera precisa el rango de tiempo desde que se entra en la región 3 hasta que ocurre el corte por exceso de velocidad (\emph{high speed cut}).
\\

La simulación se hará en Matlab Simulink y deberá presentar tanto una representación visual de la operación de las turbinas como los datos asociados a dicha operación ya sea en tablas o gráficas, según sea conveniente en cada caso. Para las características de la turbina se seguirán seguirá una línea de diseño de algún aerogenerador ya existente. Obteniendo a partir de este diseño algunos datos relevantes como la medida de las aspas (para el área de barrido del rotor). Así mismo, debido a que las condiciones del viento son diferentes en cada región geográfica, se busca que el comportamiento del viento en la simulación pueda emular con cierto grado de certeza las condiciones de algún lugar real en donde se realice la operación de turbinas eólicas. Un ejemplo de lo anterior es el tramo La Venta-La Ventosa en la región del Istmo de Tehuantepec en el estado de Oaxaca.
\\

Lo anterior con es principalmente con fines de visualización y de ejemplificación, dado que el diseño del controlador pretende ser general y no estar enfocado a un tipo específico de turbinas o diseñado para trabajar en alguna región con ciertas características climáticas en particular. 

}
\section{Trabajos relacionados}
{\parindent0pt 
El diseño de controladores para turbinas eólicas de velocidad variable se suele realizar desde dos puntos de partida: controladores con un enfoque más teórico que suelen resultar en un control más avanzado y preciso pero que funciona sobre una turbina en la que se han simplificado muchas cosas; y un enfoque más práctico que suele resultar en una turbina mejor modelada y, por lo mismo, con menor precisión al momento de operar \cite{Johnson2004}. Ambas situaciones suelen deberse al hecho de que la mayoría de las investigaciones que derivan en el diseño de controladores de velocidad concluyen en la construcción de simulaciones en las que las leyes de control expuestas funcionan bien, mas no es común que dichos controladores sean probados en turbinas reales.
\\

Para el \emph{pitch control} ha habido desde aproximaciones clásicas como lo es el uso de controladores PID \cite{Johnson2004}. Investigaciones más recientes han apuntado también al uso de controles adaptativos para el control de velocidad del rotor, aunque no han sido suficientemente probados en campo \cite{Johnson2004}. Algunas otras investigaciones han buscado alternativas a los \emph{pitch actuators} para cambiar la aerodinámica de las aspas, explorando el uso de \emph{micro-tabs} y pequeñas válvulas de aire presurizado que permite un flujo extra de aire en la superficie de las aspas y, de esta forma, cambiar el flujo de aire que cruza el área de barrido del rotor \cite{Pao2009}.
\\

Una aproximación más actual se hace en \cite{Jie2020} a partir del uso de redes neuronales y modelos predictivos tomando como entradas varios parámetros de medidos por las turbinas. A partir de esto, se utilizan modelos de aprendizaje como ELM (Extreme Learning Machine). Las salidas son tratadas como datos de referencia para el \emph{pitch control}.
\\

Se ha explorado también el uso de nuevas tecnologías que permitirían nuevas posibilidades en el diseño de controladores. Un ejemplo de esto es el uso de sensores LIDAR (del inglés \emph{Light Detection and Ranging}) para fines de medición del viento. Los sensores \emph{lidar} han sido usados exitosamente con aplicaciones meteorológicas y podrían ser interesantes de aprovechar en el control de granjas eólicas. Igualmente, turbinas más grandes han empezado a incluir actuadores independientes para cada una de las aspas, lo que permite variar el ángulo de cada una de manera independiente. Esto posibilita toda una nueva forma de realizar el control de velocidad del rotor \cite{Pao2009}. 

}


\section{Estándares de la industria}
{\parindent0pt 
El diseño de estándares para la fabricación de turbinas eólicas comienza en 1986 cuando Germanischer Lloyd publicó un conjunto de regulaciones para la certificación del diseño de turbinas eólicas que, conforme las investigaciones alrededor de este campo fueron aumentando, llevaron a la publicación en 1993 de la \emph{Regulation for the Certification of Wind Energy Conversion Systems}. Similarmente se publicaron estándares de carácter nacional en Países Bajos (NEN 6096) en 1988 y en Dinamarca (DS 472) en 1992 \cite{Burton2011}.
\\

La primera emisión de normas con un carácter verdaderamente internacional fue llevada a cabo por la \emph{International Electrothecnical Comission (IEC)} en 1988 y culminó con la publicación de las normas \emph{IEC-1400-1 Wind Turbine Generator Systems – Part 1: Safety Requirements}. Posteriores revisiones y ediciones fueron publicadas. Entre ellas la IEC-61400 publicada en 2005 y que continua actualmente en funcionamiento luego de varias revisiones \cite{Burton2011}.
}

\section{Plan de trabajo}

\textbf{Fechas de entregas:}
\begin{itemize}
    \item \emph{Capítulos I y II:} 29 de noviembre, 2023.

    \item \emph{Capitulo III:} 28 de febrero, 2024.

    \item \emph{Capítulo IV:} 01 de abril, 2024.

    \item \emph{Capítulos V y VI:} 15 de mayo, 2024. 
\end{itemize}

%!TEX root = ../main.tex
\chapter{Diseño del sistema de control}

\section{Modelo matemático del sistema}

\subsection{Subsistema mecánico}

{\parindent0pt
Este subsistema está compuesto por el eje de baja velocidad, el cual está conectado al rotor, el eje de alta velocidad 
y la caja de engranes (\emph{gearbox}). Estos tres componentes juntos conforman el tren motriz de la turbina eólica 
\cite{soltani2013}. La caja de engranes tiene como principal objetivo convertir la velocidad rotacional baja proveniente 
del rotor a una velocidad rotacional mayor requerida por el generador. Es importante destacar que la caja de engranes 
de una turbina eólica tiene una relación de engranes fija y no variable (como las que se encuentran en los automóviles) 
dado que su función únicamente es generar una mayor velocidad rotacional para alimentar al generador.
\\

Permitir que el rotor gire a mayor velocidad ayudaría a tener una transmisión con una menor relación de engranes, 
lo cual es beneficioso considerando que se trata de uno de los componentes más costosos en la turbina; sin embargo, 
esto conlleva a mayor estrés mecánico y eléctrico en el resto de los componentes, por lo que una disminución en esta 
relación de engranes debe hacerse con cuidado \cite{Burton2011}.
\\

En \cite{soltani2013} se presenta un modelo matemático para este subsistema en el cual se considera: 1) el eje de baja 
velocidad compuesto por un momento de inercia rotacional ($J_r$), un amortiguador viscoso y un resorte rotacional con 
amortiguación viscosa; 2) la caja de engranes perfectamente rígida (se asume que su flexibilidad es transferida a través 
de la deformación del eje de baja velocidad); y 3) el eje de alta velocidad compuesto por una inercia ($J_g$) y un amortiguador ($B_g$).
El modelo dinámico de esta configuración se puede simplificar a partir de incluir un modelo simplificado del tren 
motriz. La dinámica completa del sistema está dada por:

\begin{equation}
    J_t \dot{\omega}_r = \frac{1}{N_g}T_r - T_g - T_\ell
    \label{eq:torque_balance_complete}
\end{equation}

donde $J_t$ es el momento de inercia combinado del rotor y generador, $N_g$ es la relación de la caja de engranes, $T_r$ 
es el par aerodinámico recibido por el viento, $T_g$ es el par eléctrico que recibe el generador y $T_\ell$ representa el par de pérdidas.
Una forma adicional de representar la dinámica del sistema es con la ecuación:

\begin{equation}
    J \dot{\omega}_m = -f_r \omega_m + T_m - T_e
    \label{eq:torque_balance_simplified}
\end{equation}

donde $\omega_m$ representa la velocidad angular, $f_r$ es el coeficiente de fricción viscosa, $T_m$ es el par mecánico 
y $T_e$ el par eléctrico. La equivalencia se da al considerar que $T_m = T_r/N_g$, $T_e = T_g$ y que el término 
$f_r\omega_m$ modela las pérdidas mecánicas representadas por $T_\ell$ en la ecuación anterior.
En adelante, se utilizará la versión simplificada del balance de torques (\ref{eq:torque_balance_simplified}) 
puesto que permite manejar el torque de pérdidas como un valor constante en la dinámica del sistema.
\\

Es importante aclarar que en la práctica la potencia real que se puede obtener del viento es limitada y, por 
tanto, también lo es el torque aerodinámico/mecánico. En particular, la potencia extraída del viento está dada por
\footnote{Esta es una ecuación fundamental en aerodinámica de turbinas eólicas, ampliamente documentada en 
la literatura. Para una derivación detallada véase Burton et al. \cite{Burton2011}. Para aplicaciones prácticas 
y consideraciones de diseño véase Hau \cite{Hau2013}.}:
\begin{equation}
    P_a = \frac{1}{2}\rho \pi r^2 v_w^3 C_p(\lambda,\beta)
    \label{eq:aerodynamic_power}
\end{equation}

donde $\rho$ es la densidad del aire, $r$ es el radio de giro de las aspas, $v_w$ la velocidad del viento y 
$C_p(\lambda,\beta)$ es el coeficiente de potencia en términos de la relación de velocidad punta ($\lambda$) y el angulo de ataque de las aspas ($\beta$).
\\

La potencia se relaciona directamente con el torque aerodinámico mediante la ecuación:
\begin{equation}
    T_r = \frac{P_a}{\omega_r}
    \label{eq:aerodynamic_torque}
\end{equation}

En (\ref{eq:torque_balance_simplified}) utilizamos una versión simplificada del balance de torques. 
En esta ecuación se define cada torque como sigue:

\begin{equation}
    T_m = r\frac{\kappa_1}{v_w} \frac{C_p(\lambda,\beta)}{\lambda}
    \label{eq:mechanical_torque}
\end{equation}

\begin{equation}
    T_e = \frac{3}{2}p\phi i_q
    \label{eq:electrical_torque}
\end{equation}

Estas ecuaciones introducen las variables $\kappa_1$, la cual es un parámetro de diseño y un valor 
constante en la simulación, el flujo eléctrico en el generador ($\phi$) y la corriente en el eje de 
cuadratura del generador ($i_q$). Estas últimas dos variables serán tratadas
a profundidad en la sección del subsistema eléctrico.



\subsection{Coeficiente de potencia}

Durante su operación, el rotor de la turbina eólica absorbe energía de la corriente de aire, 
lo cual promueve su movimiento. El poder de absorción y las condiciones de operación del aerogenerador dependen, por tanto, 
del área efectiva de barrido ($A_R$), la velocidad del viento y los cambios que ocurren en estas 
cantidades en el campo de flujo del rotor. Durante el proceso de extracción de la potencia el flujo de aire experimenta
desaceleración axial, una desviación tangencial al momento de chocar con el área del rotor y una 
expansión del área de la sección transversal una vez que ha pasado el rotor (efecto estela) \cite{heier2014}.
\\

La potencia teórica extraida por la turbina se puede expresar como:

\begin{align}
    P_W = A_R\frac{\rho}{2}v^3
    \label{eq:theoretical_power}
\end{align}

y de acuerdo con Betz \cite{betz1926}, la potencia máxima teórica extraida sería:

\begin{align}
    P_{W_{\text{max}}} = \frac{16}{27}A_R\frac{\rho}{2}v_1^3 
    \label{eq:max_theoretical_power}
\end{align}

donde $v_1$ es la velocidad del viento antes del rotor.
\\

El cociente entre la potencia teórica extraida y la potencia en la masa de aire en movimiento, la cual se expresa como:

\begin{align}
    P_0 = A_R\frac{\rho}{2}v_1^3
\end{align}

define el coeficiente de potencia (también llamado coeficiente de rendimiento):

\begin{align}
    c_p = \frac{P_W}{P_0}
\end{align}

Se puede observar que el coeficiente de potencia consiste en el cociente entre el cubo la velocidad $v$ 
(el cual es una función de las velocidades $v_1$, $v_2$ y $v_3$, siendo estas las velocidades antes, en 
y después del rotor, respectivamente), y el cubo de la velocidad del viento antes del rotor ($v_1$). En 
la práctica, la manera más común de obtener este valor es utilizando métodos 
computacionales y artificios matemáticos más complejos (se puede encontrar un desarrollo detallado del 
método \emph{blade element} en \cite{heier2014}).
\\

Un punto importante a notar es que, de acuerdo con \cite{Burton2011}, en turbinas modernas de tres palas 
el valor máximo de $C_P$ es de solo 0.47, lo cual se aleja considerablemente del límite teórico de Betz, 
y se logra con un valor de la 
relación de velocidad punta de $\lambda^\star \approx 7$. Esto se puede ver más fácilmente en las curvas 
$C_P - \lambda$ (vease figura \ref{fig:Cp_vs_lambda}). 
\begin{figure}
    \begin{tikzpicture}
    \begin{axis}[
        title={Coeficiente de Potencia ($C_p$) vs Relación de Velocidad de Punta ($\lambda$)},
        xlabel={$\lambda$},
        ylabel={$C_p$},
        xmin=0, xmax=15,
        ymin=0, ymax=0.5,
        %xtick={0,1,2,3,4,5,6,7,8,9,10,11,12,13,14,15},
        xtick={0,5,10,15},
        ytick={0,0.1,0.2,0.3,0.4,0.5},
        grid=both,
        minor grid style={gray!25},
        major grid style={gray!50},
        width=12cm,
        height=8cm,
        every axis plot/.append style={very thick}
    ]
    
    % Definir la curva ajustada para alcanzar el máximo en lambda=7, Cp=0.47
    \addplot[smooth, domain=1:15, samples=200] {
        0.47 * (1/(1 + exp(-3*(x-3)))) * (1/(1 + 0.15*exp(0.3*(x-8)))) * (1-0.01*(x-8)^2) + 0.04
    };
    
    \end{axis}
    \end{tikzpicture}
    \caption{Curva de $\lambda$ vs $C_p$. Adaptada de \cite{Burton2011}}
    \label{fig:Cp_vs_lambda}
\end{figure}
\\


Con el objetivo de obtener un valor del coeficiente de potencia ajustado a los parámetros utilizados durante 
el modelado de la turbina se utilizará una aproximación mediante una función analítica de $c_p$.
En \cite{man1981} se desarrolla una expresión para el coeficiente de potencia como la siguiente función de 
la relación de velocidad punta y el angulo de inclinación de las palas:

\begin{align}
    C_p = c_1 \left( c_2 - c_3\beta - c_4\beta^x - c_5 \right) e^{-c_6(\lambda,\beta)}
\end{align}

Los valores de las constantes cambian según el método y modelo utilizado. A partir del procedimiento en 
\cite{amlang1992}, los valores de estas constantes se definen como:
\begin{align}
    \begin{array}{lll}
        c_1 = 0.5, &c_2 = \frac{116}{\lambda_i}, &c_3 = 0.4,
        \\
        c_4 = 0, &c_5 = 5, &c_6 = \frac{21}{\lambda_i}
    \end{array}
\end{align}

y
\begin{align}
    \frac{1}{\lambda_i} = \frac{1}{\lambda + 0.08\beta} - \frac{0.035}{\beta^3 + 1}
\end{align}

$\beta$ es el angulo de inclinación (\emph{pitch}) de las palas.

\subsection{Subsistema eléctrico}

La parte eléctrica del modelo está compuesta por: 1) el generador eléctrico (PMSG), 2) la electrónica 
de potencia y 3) la carga que será alimentada. La elección del generador 
eléctrico depende de factores como la potencia requerida para ser extraída del viento, la eficiencia 
requerida, la cantidad de mantenimiento requerida, la fiabilidad y la 
longevidad del generador en comparación con el resto de los componentes y la vida útil general de la 
turbina, entre otros. 
\\

Dos de los tipos de generadores eléctricos más usados en WECS son los Generadores de Inducción Doblemente 
Alimentados (DFIG, por sus siglas en inglés: \emph{Doubly-Fed Induction Generator}) 
y los Generadores Síncronos de Imanes Permanentes (PMSG por sus siglas en inglés: \emph{Permament Magnent 
Synchronous Generator}). Por un lado, los generadores tipo DFIG, como 
su nombre lo indica, presentan una doble alimentación de voltaje hacia el rotor y el estator. Este último 
puede ser 
conectado directamente a la red y al rotor a través de un convertidor \cite{dahiya2019}. Este tipo de 
generadores es el más usado en turbinas eólicas de velocidad variable
\cite{dahiya2019,soltani2013}.
Los PMSG, por su parte, tienen la característica de permitir el accionamiento 
directo (\emph{direct drive}) del generador, lo cual hace posible la eliminación del \emph{gearbox} de 
la mecánica del sistema. Esto se logra a partir de incluir un alto número 
de polos en el diseño del generador, lo que permite la operación a bajas velocidades del rotor \cite{soltani2013, swibki2020}.
\\

Dentro de las consideraciones generales de ambos generadores al momento de decidir entre uno u otro 
se debe mencionar: i) la mayor eficiencia de los PMSG al operar efectivamente 
en un rango más amplio de condiciones climáticas (los DFIG operan con mayor eficiencia a velocidades 
altas del viento), ii) los costos de construcción asociados a los PMSG, 
en particular por los materiales utilizados en su fabricación; y iii) la necesidad de una caja de 
cambios en los DFIG, lo que incrementa su costo e introduce el problema del 
desgaste de estos componentes con el tiempo, así como la necesidad de mantenimiento a los componentes. 
Ambos tipos de generadores están diseñados  para operar a velocidad variables y permiten el control de 
la potencia reactiva \cite{dahiya2019, swibki2020,Lei2006,feijoo2000}. 
\\

Las ecuaciones dinámicas son similares en ambos 
tipos de generadores como se puede ver en \cite{Lei2006} para DFIG y en \cite{cisneros2020} para PMSG. 
Las diferencias se dan principalmente en que DFIG se incluyen también los voltajes del rotor además del estator \cite{Lei2006}.
El posterior desarrollo matemático en esta tesis asume la elección de un Generador Síncrono de Imanes 
Permanentes (PMSG).
\\

Los generadores son sistemas eléctricos trifásicos en los que cada señal de voltaje, intensidad de corriente 
o flujo magnético se encuentra desfasada $120^\circ$ con respecto a la anterior. En este sentido, a cada fase se le 
asigna una letra para denotar sus respectivas magnitudes. Este sistema se conoce como el \textbf{marco de referencia}
\emph{abc}. A pesar de que este marco de referencia emplea las variables físicas reales del sistema ($v_{abc}$, $i_{abc}$ y $\phi_{abc}$), 
introduce una mayor complejidad al momento de diseñar el control debido a la existencia de términos dependientes del 
tiempo y cuyo comportamiendo suele ser senoidal. 
\\

Para simplificar el análisis matemático y el diseño del control se utiliza un cambio de variable que permite transformar 
las variables del sistema del marco de referencia \emph{abc} a un nuevo marco de referencia \emph{dq0} (en lo sucesivo nos referiremos 
únicamente a este marco de referencia como \emph{dq} o coordenadas \emph{dq}). En \cite{krause2013} este cambio de variable se maneja como
una transformación a un \textbf{marco de referencia arbitrario}, la cuál podría considerarse una versión genérica que agrupa a las trasformadas
específicas de Park \cite{park1929} para distintas aplicaciones. En particular, el cambio del marco de referencia \emph{abc} a \emph{dq} implica el uso de 
las transformadas de Clarke \cite{clarke1943} y Park de manera sucesiva. 
\\

En el nuevo marco de referencia las variablas de tipo \emph{dq} se conocen como variables del \textbf{eje directo} para \emph{d} y variables 
del \textbf{eje de cuadratura} para \emph{q}. Las variables \emph{0's} no serán incluidas ya que dichas magnitudes están aritméticamente relacionadas
con las variables \emph{abc} y solo se incluiran aquellas que permitan condiciones de balance. i.e. \emph{dq} al ser independientes del tiempo \cite{krause2013,krause1998}.
\\

El análisis dinámico del generador de la turbina parte de leyes fundamentales de la electricidad
\footnote{El desarrollo completo de las ecuaciones dinámicas del sistema se puede encontrar en \cite{krause2013}}. 
En particular, para los \textbf{componentes resistivos} 
del generador, la ley de Ohm establece que:
\begin{align}
    v_{abc} = R_s i_{abc}
\end{align}

donde $R_s$ es la matriz de resistencias del estator. Al transformar esta relación al marco de referencia arbitrario obtenemos:
\begin{align}
    v_{dq0} = K_sR_s(K_s)^{-1} i_{dq0}
\end{align}

Asumiendo que cada uno de los embobinados de fase del estator tienen la misma resistencia se cumple entonces que:
\begin{align}
    K_sR_s(K_s)^{-1} = R
\end{align}

Resultando en la siguiente expresión para el voltaje:
\begin{align}
    v_{dq} = R i_{dq}
\end{align}

Por su parte, los \textbf{elementos inductivos} siguen la ley de inducción electromagnética de Faraday,
\begin{align}
    v_{abc} = \frac{d}{dt}\lambda_{abc}
\end{align}

donde $\lambda_{abc}$ son los enlaces de flujo, los cuales a su vez representan el 
flujo magnético total que atraviesa un embobinado y que, en el caso de un PMSG, recibe dos aportaciones principalmente:
el flujo magnético generado en los devanados del estator y el flujo producido en los imanes permanentes en el rotor.
Al transformar a coordenadas \emph{dq} obtenemos:
\begin{align}
    v_{dq} = \omega\lambda_{dq} + \frac{d}{dt}\lambda_{dq}
\end{align}

La expresión final de los voltajes \emph{dq} resulta de la suma de los voltajes de los elementos resistivos e inductivos. 
Por tanto,
\begin{align}
    v_{dq} = Ri_{dq} + \omega\lambda_{dq} + \frac{d}{dt}\lambda_{dq}
\end{align}

En \cite{krause2013} los valores de $\lambda_{dq}$ se asignan como sigue. 
\begin{align}
    &\lambda_q = L_q i_q
    \\ 
    &\lambda_d = L_d i_d + \lambda_m
\end{align}

donde $L_q = L_{ls} + L_{mq}$, $L_d = L_{ls} + L_{md}$. i.e. en ambos casos, la inductancia total de cada eje es la suma de la inductancia
de fuga más la inductancia de magnetización (inducida por los imanes permanentes); $\lambda_m$ es el enlace de flujo de los imanes permanentes.
\\

Una manipulación algebráica permite llegar a las siguientes ecuaciones dinámicas que caracterizan el funcionamiento 
del generador utilzando coordenadas \emph{dq}:


\begin{equation}
    \begin{aligned}
        L\frac{di_q}{dt} &= -Ri_d + pLi_q\omega_m - v_d
        \\
        L\frac{di_d}{dt} &= -Ri_q - pLi_d\omega_e + p\phi\omega_e - v_q 
    \end{aligned}
    \label{eq:electrical_subsystem}
\end{equation}

donde $i_d$, $i_q$, $v_d$ y $v_q$ son los voltajes y corrientes en el eje directo y de cuadratura, $p$ es el número de polos,
$\phi$ es el flujo magnético en el generador y $R$ es el valor de la resistencia en el generador. 

\subsection{Sistema acoplado}

A partir de las ecuaciones del subsistema mecánico (\ref{eq:torque_balance_simplified}) y el subsistema eléctrico (\ref{eq:electrical_subsystem}) 
obtenemos el modelo dinámico que define el comportamiento del sistema completo de la turbina eólica:

\begin{align}
    L\frac{di_q}{dt} &= -Ri_d + pLi_q\omega_m - v_d
    \\
    L\frac{di_d}{dt} &= -Ri_q - pLi_d\omega_e + p\phi\omega_e - v_q 
    \\
    J \dot{\omega}_m &= -f_r \omega_m + T_m - T_e
    \label{eq:full_system}
\end{align}

Donde
\begin{align}
    T_m &:=T_m(\lambda,\beta)= r\frac{\kappa_1}{v_w} \frac{C_p(\lambda,\beta)}{\lambda}
    \\
    T_e  &:= T_e(i_q)= \frac{3}{2}p\phi i_q
    \\
    \lambda &:=\lambda(\omega_m)= \frac{r\omega_m}{v_w}
    \label{eq:definitions}
\end{align}

Los voltajes $v_d$ y $v_q$ se pueden manipular de forma libre, es decir, corresponden a las variables de control.

\section{Diseño del control de torque}

\subsection{Objetivos de control}

En turbinas éolicas de velocidad variable obtenemos una mayor eficiencia del sistema al reducir las pérdidas
del generador para una carga dada. Dentro del marco de referencia \emph{dq} la corriente $i_q$ se emplea para
inducir el torque en el generador. Esto se puede apreciar en la ecuación del torque eléctrico (\ref{eq:electrical_torque})
en la cual la corriente $i_d$ no tiene ninguna contribución. Por lo anterior, y para minimizar pérdidas, 
buscamos llevar la corriente del eje directo a un valor de $i_d = 0$ \cite{chinchilla2006}.
Por su parte, el valor de la corriente $i_q$ deberá ser controlado con el objetivo de maximizar la extracción 
de potencia del viento. Lo anterior tiene como objetivo final la regulación de la velocidad rotacional del 
rotor de la turbina. 
\\

Previamente se introdujo el coeficiente de potencia. En dicha función se cumple que
para cada velocidad del viento existe una velocidad angular óptima que maximiza $C_p(\lambda,\beta)$ \cite{Leidhold2002,nak2013}.
Dicha velocidad angular óptima se conoce como \textbf{punto de máxima potencia} o MPP (por sus siglas en inglés: 
\emph{Maximum Power Point}) y al método o estrategia de control para encontrar este punto se le conoce como
\textbf{seguimiento del punto de máxima potencia} o MPPT (Por sus siglas en inglés: \emph{Maximum
Power Point Tracking}). Aunque existen varios métodos para el MPPT, en \cite{nak2013} se 
clasifican en tres principales categorías: control de la relación de velocidad punta (TSR), búsqueda 
\emph{Hill-climbing} (HCS), también conocida como \emph{Perturb \& Observe} (P\&O); y, por último, control de 
retroalimentación de la señal de potencia (PSF).
\\

Se considerará en el diseño del control del torque un algoritmo P\&O (o HCS) para el control de la velocidad óptima del
rotor. La estrategia principal en este tipo de control consiste en tomar muestras de la salida de potencia 
continuamente y compararlas con la salida inmediata anterior. Dependiendo de si la salida actual es mayor o menor
se incrementa o decremente el torque eléctrico del generador \cite{wang2004,buehring1981}.
\\

Una de las ventajas que ofrece un algoritmo HCS consiste en ser, de entre los tres tipos de estrategias
de control, el único que no requiere un conocimiento previo del sistema WECS y que es independiente de 
las características del viento, del generador y de la turbina en general. Lo anterior implica que el MPPT
se puede realizar en conjunto con otras estrategias de control para WECS como el \emph{pitch control} 
\cite{kazmi2011}. 
\\

Por otra parte, los algoritmos HCS presentan dos desventajas que pueden estar inducidas por alteraciones 
en la velocidad del viento en periodos cortos de tiempo o a un tamaño de paso del algoritmo inadecuado 
\cite{kazmi2011,xia2013}. Sin ambargo, estos problemas serán tratados con mayor detalle en la sección de
simulación dado que, como se mencionó, su Implementación es independiente de la dinámica del generador y 
del resto de la turbina. 
\\

Por tanto, los objetivos de control se pueden resumir en:
\begin{enumerate}
    \item regular la corriente del eje directo en $i_d = i_d^\star = 0$ para minimizar pérdidas,
    \item regular la velocidad angular $\omega_m$ a un valor constante positivo que corresponta a 
        la velocidad óptima del MPPT.
\end{enumerate}

\subsection{Estrategia de control}

El sistema de control propuesto combina distintas estrategias para regulación de la velocidad del rotor.
Para empezar, se regula el torque del generador mediante un control PI cuyo objetivo es mantener las variables
del PMSG en coordenadas \emph{dq} en un régimen adecuado para la máxima extracción de potencia del viento. Para
esto, se utiliza como referencia el valor de la velocidad rotacional y un algoritmo de seguimiento del punto de
máxima potencia (MPPT) para ajustar la velocidad rotacional óptima a la cual se quiere seguir.
\\

Con base en la sección anterior, en particular las ecuaciones (\ref{eq:full_system}) que describen el 
sistema completo, nos interesa encontrar las ecuaciones que definen las variables de control $v_d$ y 
$v_q$. La estrategia utilizada para lograr esto es buscar un patrón de ``cancelación'' de términos 
no lineales combinado con la inserción de términos del control PI. De este modo, partiendo del modelo 
dinámico, se tiene para $v_d$:
\begin{align}
    v_d = pL i_q \omega_m + u_{\tt pi}^d \label{valUd},
\end{align}

donde $u^d_{\tt pi}$ es la salida del controlador PI definida, por tanto, como:
\begin{equation}
\begin{aligned}
    u_{\tt pi}^d &= k_{P1}(i_d-i_{d}^{\star}) + k_{I1}\xi_1\\
    &=k_{P1}i_d + k_{I1}\xi_1\\
    \dot{\xi_1} &= i_d-i_d^{\star}= i_d.
\end{aligned}
\label{eqUd}
\end{equation}

De manera similar para $v_q$:
\begin{align}
v_q = -pL i_d \omega_m + p\phi \omega_m + u_{\tt pi }^q,
\label{eqnVq}
\end{align}

donde $u_{\tt pi}^q$ corresponde al controlador PI definido por las siguientes ecuaciones:
\begin{equation}
\begin{aligned}
u^q_{\tt pi} &= k_{P2}(i_q - i_{q}^{ref}) + k_{I2}\xi_2 \\
\dot{\xi}_2 &= i_q - i_{q}^{ref},
\end{aligned}
\label{PI2}
\end{equation}

con la señal $i_q^{ref}$ definida como :
\begin{equation}\label{xi}
\begin{aligned}
i_{q}^{ref} &= k_{P3}(\omega_m - \omega_m^\star) + k_{I3}\xi_3 \\
\dot{\xi}_3 &= \omega_m - \omega_m^\star.
\end{aligned}
%\label{PI3}
\end{equation}

La estructura completa del control está definida por tres lazos de retroalimentación caracterizados
como sigue:

\begin{enumerate}
    \item \textbf{control de la corriente $i_d$}: \begin{itemize}
        \item entrada: la corriente $i_d$ medida que se busca mantener en 0,
        \item salida: el valor de $u_d$ para que $v_d$ compense las variaciones en $i_d$;
    \end{itemize}
    \item \textbf{control de la corriente $i_q$}: \begin{itemize}
        \item entrada: error de la corriente en el eje de cuadratura $(i_q - i_q^{ref})$,
        \item salida: el valor de $u_q$ para $v_q$;
    \end{itemize}
    \item \textbf{control de velocidad}: \begin{itemize}
        \item entrada: el error de la velocidad del rotor $(\omega_m - \omega_m^\star)$;
        \item salida: la señal de referencia de la corriente $i_q^{ref}$.
    \end{itemize}
\end{enumerate}

La sustitución de \eqref{eqUd} en \eqref{valUd} y, la sustitución de la expresión resultante en la primera
ecuación de \eqref{eq:electrical_subsystem} da lugar a las ecuaciones de $\Sigma_1$. Esto es:
\begin{align}
    L\dot{i_d} &= -(R+k_{P1})i_d - k_{I1}\xi_1\\
    \dot{\xi_1} &= i_d,
\end{align}

o, equivalentemente, % en representación de estados,
\begin{align}\label{eq:sigma_1}
    \mathbf{\Sigma_1:}\;\;\: \dot x= & \underbrace{\begin{bmatrix}0&1\\  -\frac{1}{L}k_{I1} & -\frac{1}{L}(R+k_{P1}) 
    \end{bmatrix}}_{A_1}x
 \end{align}

 con $x=\mathrm{col}(\xi_1,i_d)$. Claramente este subsistema corresponde a un sistema lineal e invariante en el tiempo. \\
Para el segundo subsistema, sustituyendo \eqref{eqnVq}, \eqref{PI2} y \eqref{xi} en la segunda ecuación de  \eqref{eq:electrical_subsystem}  resulta en la siguiente dinámica
\begin{align*}
    L\dot{i_q} =& -(R+k_{P2})i_q + k_{P2}i_q^{ref} \\
    =&-(R+k_{P2})i_q  + k_{P2}k_{P3}(\omega_m - \omega_m^\star) + k_{P2}k_{I3}\xi_3 - k_{I2}\xi_2
\end{align*}

Uniendo la ecuación anterior con las de la dinámica de $\xi_2$ y de $\xi_3$ (descritas en  \eqref{PI2} y 
\eqref{xi}) resulta en la siguiente representación en espacio de estados
\begin{equation}\label{eq:sigma_2}
\resizebox{\textwidth}{!}{$
\mathbf{\Sigma_2:}\;\;\begin{aligned}
	\dot z=& \begin{bmatrix}
    \dot{\xi}_2 \\
    \dot{\xi}_3 \\
    \dot{\omega}_m \\
    \dot{i}_q
    \end{bmatrix} = \begin{bmatrix}
    z_4 - (k_{P3}(z_3 - \omega^*_m) + k_{I3}z_2) \\
    z_3 - \omega^*_m \\
    -\frac{f_r}{J}z_3 + \frac{r\kappa_1}{Jv_w}\frac{C_{p,0}(\lambda)}{\lambda} - \frac{3}{2J}p\phi z_4 \\
    -\frac{R+k_{P2}}{L}z_4 + \frac{k_{P2}k_{P3}}{L}(z_3 - \omega^*_m) + \frac{k_{P2}k_{I3}}{L}z_2 - \frac{k_{I2}}{L}z_1
    \end{bmatrix}
\end{aligned}=:f(z)
$}
\end{equation}
donde $z=\mathrm{col}(\xi_2,\xi_3,\omega_m,i_q)$ y $C_{p,0}(\lambda):=C_{p}(\lambda,0)$. Asumiendo la velocidad del 
viento constante,\footnote{Se aclara al lector que esta suposición se asume solamente para el propósito del control 
propuesto. Como se verá en las simulaciones, el hacer esto resulta en un control sencillo con un desempeño aceptable 
cuando el sistema es sometido a perfiles de viento realistas.} el sistema $\Sigma_2$ anterior es un sistema invariante del tiempo. 

% --------------- LEMA ------------------
La discusión anterior se resume en el siguiente lema.\\

\begin{lemma}
    Considere el sistema WECS \eqref{eq:full_system}-\eqref{eq:definitions} en lazo cerrado con \eqref{valUd}-\eqref{xi}. El sistema 
    resultante se compone de dos subsistemas desacoplados, que denominaremos $\Sigma_1$ y $\Sigma_2$, 
    cuyas ecuaciones son las siguientes:
    \begin{align*}
        \Sigma_1: \dot x= & \begin{bmatrix}0&1\\  -\frac{1}{L}k_{I1} & -\frac{1}{L}(R+k_{P1}) 
            \end{bmatrix}\\
        \Sigma_2: \dot z=& \begin{bmatrix}
            z_4 - (k_{P3}(z_3 - \omega^*_m) + k_{I3}z_2) \\
            z_3 - \omega^*_m \\
            -\frac{f_r}{J}z_3 + \frac{r\kappa_1}{Jv_w}\frac{C_{p,0}(\lambda)}{\lambda} - \frac{3}{2J}p\phi z_4 \\
            -\frac{R+k_{P2}}{L}z_4 + \frac{k_{P2}k_{P3}}{L}(z_3 - \omega^*_m) + \frac{k_{P2}k_{I3}}{L}z_2 - \frac{k_{I2}}{L}z_1
            \end{bmatrix}
    \end{align*}
    \hfill{$\triangle\triangle\triangle$}
\end{lemma}

A continuación, se linealizará el sistema alrededor del punto de equilibrio que cumpla con los objetivos 
de control. Luego, se obtendrán las condiciones en las ganancias $k_{P}$s y $K_I$s que hacen que esa 
linealización tenga valores propios en el semiplano complejo izquierdo. De acuerdo con el Método Indirecto 
de Lyapunov, esto garantiza que el punto de equilibrio deseado sea exponencialmente estable.

\subsubsection*{Puntos de equilibrio de $\Sigma_1$ y $\Sigma_2$}

Como parte del análisis de estabilidad, primero se obtienen los puntos de equilibrio de ambos subsistemas. 
Estos quedan enunciados en los siguientes lemas.

\begin{lemma}
    El sistema $\Sigma_1$ tiene un solo punto de equilibrio en $$x=\bar{x}:=0$$.
    \hfill{$\triangle\triangle\triangle$}

    \begin{proof}
        Al igualar $\dot x=0$ en (1.37), resulta en la siguientes ecuaciones
        \begin{align*}
            x_2=&0\\
            -(R+k_{P1})x_2-k_{I1}x_1  =&0 	
        \end{align*}
        Se concluye que $x_2=0$ es un punto de equibrio. Reemplazando lo anterior en la segunda ecuación queda
        $$k_{I1}x_1  =0,$$
        de donde se concluye que  tambien  $x_1=0$ es el valor de equilibrio para $x_1$.
    \end{proof}
\end{lemma}

\begin{lemma}
    El punto $\Sigma_2$ tiene un solo punto de equilibrio en 
	\begin{align}\label{barz}
	z=\bar z:=
		\begin{bmatrix}
			-\frac{2R}{3p\phi k_{I2}}(T^*_m - f_r\omega^*_m) \\[0.3cm]
			\frac{2}{3p\phi k_{I3}}(T^*_m - f_r\omega^*_m) \\[0.3cm]
			\omega^*_m \\[0.3cm]
			\frac{2}{3p\phi}(T^*_m - f_r\omega^*_m)
	\end{bmatrix}
	\end{align}
    donde $T^*_m := r\frac{\kappa_1 v_w C_{p,0}(\lambda^*)}{\lambda^*}$ es el torque mecánico en el 
    equilibrio, $\lambda*:=\frac{r\omega_m^\star}{v_w}$  y $\omega_m^\star$ es la referencia de la velocidad 
    angular  $\omega_m$.

    \hfill{$\triangle\triangle\triangle$}

    \begin{proof}
        Para encontrar los puntos de equilibrio, hacemos $\dot z=0$ en \eqref{eq:sigma_2}. Esto resulta el siguiente sistema de ecuaciones:

        \begin{align}
        z_4 - [k_{P3}(z_3 - \omega^*_m) + k_{I3}z_2] &= 0 \label{eq:ss1} \\
        z_3 - \omega^*_m &= 0 \label{eq:ss2} \\
        -\frac{f}{J}z_3 + \frac{r\kappa_1}{Jv_w}\frac{C_{p,0}(\lambda)}{\lambda} - \frac{3}{2J}p\phi z_4 &= 0 \label{eq:ss3} \\
        -\frac{R+k_{P2}}{L}z_4 + \frac{k_{P2}k_{P3}}{L}(z_3 - \omega^*_m) + \frac{k_{P2}k_{I3}}{L}z_2 - \frac{k_{I2}}{L}z_1 &= 0 \label{eq:ss4}
        \end{align}

        Procedemos a resolver el sistema paso a paso:

        \begin{enumerate}
        \item De la ecuación \eqref{eq:ss2} obtenemos directamente:
        \begin{equation}
        \bar{z}_3 = \omega^*_m \label{eq:eq1}
        \end{equation}

        \item Sustituyendo \eqref{eq:eq1} en \eqref{eq:ss1}:
        \begin{align}
        \bar{z}_4 - k_{I3}\bar{z}_2 &= 0 \label{eq:eq2}
        \end{align}

        \item De la ecuación \eqref{eq:ss3}, usando \eqref{eq:eq1} y $\lambda^* = \frac{r\omega^*_m}{v_w}$:
        \begin{equation}
        \bar{z}_4 = \frac{2}{3p\phi}(T^*_m - f\omega^*_m) \label{eq:eq3}
        \end{equation}
        %donde $T^*_m = r\frac{\kappa_1 v_w C_p(\lambda^*, 0)}{\lambda^*}$ es el torque mecánico en el equilibrio y $\omega_m^\star$ es la referencia de la velocidad angular de $\omega_m$.

        \item Usando \eqref{eq:eq2} y \eqref{eq:eq3}:
        \begin{equation}
        \bar{z}_2 = \frac{1}{k_{I3}}\bar{z}_4 = \frac{2}{3p\phi k_{I3}}(T^*_m - f\omega^*_m) \label{eq:eq4}
        \end{equation}

        \item Finalmente, de la ecuación \eqref{eq:ss4}, usando las ecuaciones anteriores:
        \begin{equation}
        \bar{z}_1 = -\frac{R}{k_{I2}}\bar{z}_4 = -\frac{2R}{3p\phi k_{I2}}(T^*_m - f\omega^*_m) \label{eq:eq5}
        \end{equation}
        \end{enumerate}

        Por lo tanto, el punto de equilibrio es el que se muestra en el enunciado del lema.
    \end{proof}
\end{lemma}

\subsection{Análisis de estabilidad de $\Sigma_1$ y $\Sigma_2$ }


Una vez que se encuentran los puntos de equilibrio, en esta subsección se analizará la estabilidad de 
los mismos. Puesto que los subsistemas están desacoplados, la estabilidad de cada uno se puede estudiar 
por separado.\\

\begin{lemma}
    Considere el sistema $\Sigma_1$ con $k_{P1}>-R$ y $k_{I1}>0$. Entonces, su punto de equilibrio 
    $x=\bar x=0$ es exponencialmente, globalmente estable.

    \hfill{$\triangle\triangle\triangle$}

    \begin{proof}
        El polinomio característico de  $A_1$  en (\ref{eq:sigma_1}) es el siguiente:
        \begin{align}\label{p1}
            p_1(s)=s^2 + \frac{k_{P1}+R}{L}s + \frac{k_{I1}}{L}.
        \end{align}

        Luego, por el criterio de estabilida de Routh, la condición necesaria y suficiente para que un polinomio de grado 
        dos tenga sus raices en el semiplano izquierdo es que sus coeficientes sean positivos. Por tanto, tomando 
        en cuenta que $L>0$, tenemos las siguientes relaciones
        \begin{align}
            k_{P1} +R > 0  \Longleftrightarrow k_{P1} &> -R\\
            k_{I1} &> 0 
        \end{align}
    \end{proof}
\end{lemma}

% -------------------\\
% De lo anterior, los valores proporcional e integral $k_{P1} = 0$ y $k_{I1} = -1$ garantizan la estabilidad de $\Sigma_1$.\\
% -------------------

Por otro lado, para estudiar la estabilidad del subsistema $\Sigma_2$, se procederá a linealizarlo.
Para tal motivo se calcula el Jacobiano y se evalúa en  el punto de equilibrio. Esto es, el sistema 
linealizado tiene la forma:
\begin{align}\label{dz}
\Sigma_{2}^\mathrm{lin}:\;\;\;\;\dot{\delta z} =A_2 \delta z
\end{align}
donde $\delta z$ representa a la variable del sistema linealealizado,  
\begin{equation}
A_2 = \left.\frac{\partial f}{\partial z}\right|_{z=\bar{z}} = \begin{bmatrix}
0 & -k_{I3} & -k_{P3} & 1 \\
0 & 0 & 1 & 0 \\
0 & 0 & -\frac{f_r}{J} + {\gamma \over J} & -\frac{3}{2J}p\phi \\
-\frac{k_{I2}}{L} & \frac{k_{P2}k_{I3}}{L} & \frac{k_{P2}k_{P3}}{L} & -\frac{R+k_{P2}}{L}
\end{bmatrix} \label{eq:A2_symbolic}
\end{equation}

y usando la regla de la cadena:
\begin{align}
\gamma :=&\left. \frac{r\kappa_1}{v_w} \frac{\partial }{\partial \lambda}\left[\frac{C_{p,0}}{\lambda }\right]\right|_{\omega_m=\omega^*_m,\lambda=\lambda^\star}\nonumber\\
=&\left. \frac{r\kappa_1}{v_w}\frac{\partial\lambda}{\partial \omega_m}\left[{1\over \lambda}\frac{\partial C_{p,0}}{\partial\lambda\nonumber }-\frac{1}{\lambda^2}C_{p,0}\right]\right|_{\omega_m=\omega^*_m,\lambda=\lambda^\star}\\
=&  \frac{r^2\kappa_1}{v_w^2}{1\over \lambda^\star}\left[\left.\frac{\partial C_{p,0}}{\partial \lambda}\right|_{\lambda=\lambda^\star}-{1\over \lambda^\star} C_{p,0}(\lambda^\star) \right]\nonumber \\
=& - \frac{r^2\kappa_1}{v_w^2}{1\over (\lambda^\star)^2}  C_{p,0}(\lambda^\star)
\label{eq:gamma}
\end{align}

Puesto que los parámetros son positivos y $C_{p,0}>0$, entonces $\gamma<0$. A partir de lo anterior, 
es posible enunciar el siguiente lema. Así mismo, es posible concluir que la entrada (3,3) de la 
matriz $A_2$ toma valores negativos, esto es, $A_{2}^{(3,3)}<0$.\\

\begin{lemma}
    Considere el sistema $\Sigma_2^{\mathrm{lin}}$ en \eqref{dz} que consiste en la linealización 
    de $\Sigma_2$. El punto de equilibrio $z=\bar z$, con la constante $\bar z$  definida en 
    \eqref{barz}, es exponencialmente estable si y solo si .... 

    \begin{proof}
        La estabilidad del sistema  \eqref{dz} se puede inferir de la estabilidad de la matriz%\footnote{\textcolor{blue}{Note que la matriz $\bar A_2$ se puede obtener a partir de un cambio de variables de \eqref{barz}. Por ser muy sencillo, el procedimiento se omitirá.}}
        $$\bar A_2:=\begin{bmatrix}
            0 & -k_{I3} & -k_{P3} & 1 \\
            0 & 0 & 1 & 0 \\
            0 & 0 & -\bar \gamma_1  & -\bar\gamma_2 \\
            -K_{I2}' & k_{P2}'k_{I3} & k_{P2}'k_{\text P3}& -(\frac{R}{L}+k_{P2}')
        \end{bmatrix}$$
        donde $\bar\gamma_1:=\frac{f_r-\gamma}{J}>0,$ $\bar \gamma_2:=\frac{3}{2J}p\phi>0$, $K_{I2}'=\frac{k_{I2}}{L}$, $K_{P2}'=\frac{K_{P2}}{L}$.
    \end{proof}
\end{lemma}

El polinomio característico de la matriz $\bar{A_2}$ es 

\begin{equation}\label{eq:poly_sigma_2}
    \begin{split}
        p_2(s) =& s^4 + \left(\frac{R}{L} + \bar{\gamma_1} + k_{\textrm P2}'\right)s^3 + \ldots 
        \\ 
        &\ldots + \left( \frac{R\bar{\gamma_1}}{L} + k_{\textrm I2}' + \bar{\gamma_1}k_{\textrm P2}' + \bar{\gamma_2}k_{\textrm P2}'k_{\textrm P3} \right)s^2 + \ldots 
        \\
        &\ldots + \left( \bar{\gamma_1}k_{\textrm I2}' + \bar{\gamma_2}k_{\textrm I2}'k_{\textrm P3} + \bar{\gamma_2}k_{\textrm I3}k_{\textrm P2}' \right)s + \ldots
        \\
        &\ldots + \left( \bar{\gamma_2}k_{\textrm I2}'k_{\textrm I3} \right)
    \end{split}
\end{equation}

El polinomio característico tiene la forma :
\begin{equation}\label{eq:poly_sigma_2_form}
    p_2(s) = a_4s^4 + a_3s^3 +a_2s^2 a_1s + a_0
\end{equation}

La tabla de Routh-Hurwitz para un polinomio de cuarto orden requiere que la primera columna tenga 
todos sus elementos positivos. La tabla tiene la siguiente estructura:

\begin{equation}
\begin{array}{c|ccc}
s^4 & a_4 & a_2 & a_0 \\
s^3 & a_3 & a_1 & \textcolor{white}{0} \\
s^2 & b_1 & b_2 & \textcolor{white}{0} \\
s^1 & c_1 & \textcolor{white}{0} & \textcolor{white}{0} \\
s^0 & d_1 & \textcolor{white}{0} & \textcolor{white}{0}
\end{array} \label{eq:rh_table}
\end{equation}

donde:
\begin{align}
b_1 &= \frac{a_3a_2 - a_4a_1}{a_3} \label{eq:b1} \\
b_2 &= \frac{a_3a_0 - a_4 \cdot 0}{a_3} = \frac{a_3a_0}{a_3} = a_0 \label{eq:b2} \\
c_1 &= \frac{b_1a_1 - a_3b_2}{b_1} = \frac{b_1a_1 - a_3a_0}{b_1} \label{eq:c1} \\
d_1 &= a_0 \label{eq:d1}
\end{align}

Para garantizar estabilidad, necesitamos que todos los coeficientes de $p_2(s)$ sean positivos y que se cumpla:

\begin{align*}
b_1 &= \frac{a_3a_2 - a_4a_1}{a_3} > 0 \\
c_1 &= \frac{b_1a_1 - a_3a_0}{b_1} > 0 \\
\end{align*}


El coefiente $a_4 = 1$ es constante y positivo.\\ 

Para el coeficiente $a_3$ necesitamos que se cumpla:
\begin{equation} \label{eq: cond_a3}
    \begin{split}
    & \frac{R}{L} + \bar{\gamma_1} + k_{\textrm P2}' > 0
    \\
    \Longleftrightarrow\;\; & k_{\textrm P2}' > - \left(\frac{R}{L} + \bar{\gamma_1}\right)
    \\
    \Longleftrightarrow\;\; & k_{\textrm P2} > -(R + L\bar{\gamma_1}) \text{ sabiendo que } L > 0
    \end{split}
\end{equation}

Para el coeficiente $a_2$ la condición es:
\begin{align} \label{eq: cond_a2}
    & \frac{R\bar{\gamma_1}}{L} + k_{\textrm I2}' + \bar{\gamma_1}k_{\textrm P2}' + \bar{\gamma_2}k_{\textrm P2}'k_{\textrm P3} > 0
\end{align}

Para el coeficiente $a_1$ la condición es:
\begin{align} \label{eq: cond_a1}
    & \bar{\gamma_1}k_{\textrm I2}' + \bar{\gamma_2}k_{\textrm I2}'k_{\textrm P3} + \bar{\gamma_2}k_{\textrm I3}k_{\textrm P2}' > 0
\end{align}

Para el coeficiente $a_0$ la condición es:
\begin{align*}
    & \bar{\gamma_2}k_{\textrm I2}'k_{\textrm I3} > 0
    \\
    \Longleftrightarrow\;\; & k_{\textrm I2}' > 0; k_{\textrm I3} > 0
\end{align*}

Para que se cumpla que $b_1 > 0$ necesitamos:
\begin{align*}
    b_1 = \frac{a_3a_2 - a_4a_1}{a_3} > 0
\end{align*}
Como $a_4 = 1$
\begin{align}
    \Rightarrow\;\; & b_1 = \frac{a_3a_2 - a_1}{a_3} = a_2 - \frac{a_1}{a_3} > 0 \Rightarrow\;\; \varrho > 0 \label{eq: cond_b1}
\end{align}

con 
\begin{equation*}
    \varrho := k_{\textrm P2}' \left[\bar{\gamma_1} + \bar{\gamma_2}k_{\textrm P3}\right] + \frac{R}{L}\bar{\gamma_1} + k_{\textrm I2}' 
    - \frac{k_{\textrm I2}'\left[\bar{\gamma_1} + \bar{\gamma_2}k_{\textrm P3}\right] + \bar{\gamma_2}k_{\textrm I3}k_{\textrm P2}'}{\frac{R}{L} + \bar{\gamma_1} + k_{\textrm P2}'}
\end{equation*}

Para que se cumpla que $c_1 > 0$ necesitamos:
\begin{align}
    & c_1 = \frac{b_1a_1 - a_3b_2}{b_1} = \frac{b_1a_1 - a_3a_0}{b_1} = a_1 - \frac{a_3a_0}{b_1} > 0
    \\
    \Rightarrow\;\; & \bar{\gamma_2}\left[k_{\textrm I2}'k_{\textrm P3} + k_{\textrm I3}k_{\textrm P2}'\right] + \bar{\gamma_1}k_{\textrm I2}'
    - \frac{\bar{\gamma_2}k_{\textrm I2}'k_{\textrm I3}\left(\frac{R}{L} + \bar{\gamma_1} + k_{\textrm P2}'\right)}{\varrho} > 0\label{eq: cond_c1}
\end{align}

El siguiente sencillo algoritmo sirve como pauta para encontrar un conjunto de ganancias que satisfagan las condiciones \eqref{eq: cond_a3} - \eqref{eq: cond_c1}.\\

\begin{tabular}{l}
\hline
\textbf{Algoritmo.} Encontrar encontrar ganancias $k_{P2}',k_{P3},k_{I2}',k_{i3}$\\
\hline
\textbf{Paso 1}. Asignar valores positivos a $k_{P3},k_{I3}$.\\
\textbf{Paso 2}. Fijar $k_{I2}'=1/k_{P2}'$\\
\textbf{Paso 3.} Encontrar $k_{P2}'>0$ suficientemente grande tal que satisfaga $b_1$ y $c_1$\\\hline
\end{tabular}
\vspace{1em}

Note que, de acuerdo con el Paso 1 y Paso 2,   haciendo $k_{P3}=\bar{k}_{P3}>0$, $k_{I3}=\bar{k}_{I3}>0$ y $k_{I2}'=\frac{1}{k_{P2}'}$, la desigualdad \eqref{eq: cond_b1} es equivalente a:
\begin{align*}
&\varrho (k_{P2}',\bar k_{P3},\frac{1}{k_{P2}'},\bar k_{I3}) = \ldots \\
&\ldots k_{P2}'\big [ \bar\gamma_1 + \bar\gamma_2 \bar{k}_{P3} \big ]+\frac{R}{L}\bar\gamma_1  + \frac{1}{k_{P2}'} - \frac{\frac{1}{k_{P2}'^2}\big [ \bar\gamma_2\bar{k}_{P3}+\bar\gamma_1 \big] + \bar\gamma_2 \bar{k}_{I3}}{1+\frac{1}{L}\frac{R}{k_{P2}'}+\frac{\bar\gamma_1}{k_{P2}'}}>0
\end{align*}

%\textcolor{red}{Podrías expandir la desigualdad anterior y ponerla como polinomio de $k_{p2}$?}

Note de la ecuación anterior que a medida que $k_{P2}$ toma valores más grandes, la parte positiva domina por o que \eqref{eq: cond_b1} se satisface. Entre más grande es $k_{p2}$, más grande es $\varrho$. Esto implica que la cuarta condición \eqref{eq: cond_c1}:
\begin{align}
\bar\gamma_2 \big[  \frac{1}{k_{P2}'}\bar{k}_{P3} + \bar{k}_{I3}k_{P2}'\big] + \frac{\bar\gamma_1}{k_{P2}'}- \frac{\frac{1}{k_{P2}'}(\frac{R}{L}+k_{P2}'+\bar\gamma_1)\bar\gamma_2\bar{k}_{I3}}{\varrho (k_{P2}',\bar k_{P3},\frac{1}{k_{P2}'},\bar k_{I3}) }>&0
\end{align}

toma valores positivos cada vez mayores, a medida que crece $k_{P2}$. Esto es, existe un valor de $k_{P2}$ suficientemente grande que garantiza que las condiciones de Routh-Hurwitz se satisfagan. De la discusión anterior se puede afirmar  la existencia de solución siguiente.\\

\begin{lemma}
    La condiciones \eqref{eq: cond_a3}-\eqref{eq: cond_c1} tienen solución. 
\end{lemma}



%_---------------

% \textcolor{blue}{Comentario: Puedes repetir el análisis de RH con la matriz $\bar A_2$? Las k's, R, fr y $\gamma$ son constantes. No hay por que entrar en detalle en cómo está definida la funcion Cp para realizar este análisis. }
% \textcolor{blue}{
% Es decir, realizar lo que tienes en (1.65)-(1.69) pero escribiendo de forma simbóilica los parámetros. En el  polinomio caracteristicco deben aparecer los parámetros anteriores.}


% \hfill{$\square$}

% ---------------\\
% \textcolor{blue}{Comentario: No es necesario definir los valores de los parámetros. Eso se puede hacer después. Tal vez en la parte de simulación.}\\
% -----------------

% -------------------\\

% La matriz $A_2$ con los valores numéricos es:

% \begin{equation}
% \resizebox{\textwidth}{!}{$
% A_2 = 
% \begin{bmatrix}
% 0 & -k_{I3} & -k_{P3} & 1 \\
% 0 & 0 & 1 & 0 \\
% 0 & 0 & -3.86\times10^{-5} - 0.0015 & -1.5328 \\
% -281.69k_{I2} & 281.69k_{P2}k_{I3} & 281.69k_{P2}k_{P3} & -281.69k_{P2} - 103.55
% \end{bmatrix}
% $}
% \label{eq:A2_numeric_final}
% \end{equation}

% Y su polinomio característico:
% \begin{multline}
% p_2(s) = s^4 + (281.69k_{P2} + 103.55)s^3 + \\ 
% (281.69k_{I2} + 431.76k_{P2}k_{P3} - 0.0015k_{P2} - 0.0004)s^2 + \\
% (431.76k_{I2}k_{P3} - 0.0015k_{I2} + 431.76k_{P2}k_{I3})s + 431.76k_{I2}k_{I3} \label{eq:char_poly}
% \end{multline}

% El polinomio característico tiene la forma:
% \begin{equation}
% p_2(s) = a_3s^4 + a_2s^3 + a_1s^2 + a_0s + a_4 \label{eq:poly_rh}
% \end{equation}

% donde:
% \begin{align}
% a_3 &= 1 \label{eq:a3} \\
% a_2 &= 281.69k_{P2} + 103.55 \label{eq:a2} \\
% a_1 &= 281.69k_{I2} + 431.76k_{P2}k_{P3} - 0.0015k_{P2} - 0.0004 \label{eq:a1} \\
% a_0 &= 431.76k_{I2}k_{P3} - 0.0015k_{I2} + 431.76k_{P2}k_{I3} \label{eq:a0} \\
% a_4 &= 431.76k_{I2}k_{I3} \label{eq:a4}
% \end{align}


% donde:
% \begin{align}
% b_1 &= \frac{a_2a_1 - a_0}{a_2} \label{eq:b1} \\
% b_2 &= \frac{a_2a_4 - 0}{a_2} = a_4 \label{eq:b2} \\
% c_1 &= \frac{b_1a_0 - a_4}{b_1} \label{eq:c1} \\
% d_1 &= a_4 \label{eq:d1}
% \end{align}

% Para garantizar estabilidad, necesitamos:

% \begin{align}
% a_2 &> 0 \label{eq:cond1} \\
% a_1 &> 0 \label{eq:cond2} \\
% a_0 &> 0 \label{eq:cond3} \\
% a_4 &> 0 \label{eq:cond4} \\
% b_1 &> 0 \label{eq:cond5} \\
% c_1 &> 0 \label{eq:cond6}
% \end{align}

% De \eqref{eq:cond1}:
% \begin{equation}
% 281.69k_{P2} + 103.55 > 0 \implies k_{P2} > -0.3676 \label{eq:kp2_cond}
% \end{equation}

% De \eqref{eq:cond4}:
% \begin{equation}
% 431.76k_{I2}k_{I3} > 0 \implies k_{I2}k_{I3} > 0 \label{eq:ki_cond}
% \end{equation}

% Para simplificar el análisis, podemos hacer:
% \begin{equation}
% k_{P2} = 0, \quad k_{I2} = 1, \quad k_{I3} = 1 \label{eq:k_values}
% \end{equation}

% Sustituyendo estos valores en \eqref{eq:cond2} y \eqref{eq:cond3}:
% \begin{align}
% 281.69 + 431.76k_{P3} - 0.0004 &> 0 \label{eq:kp3_cond1} \\
% 431.76k_{P3} - 0.0015 + 0 &> 0 \label{eq:kp3_cond2}
% \end{align}

% De donde obtenemos:
% \begin{equation}
% k_{P3} > 3.48\times10^{-6} \label{eq:kp3_final_cond}
% \end{equation}

% Por lo tanto, un conjunto de valores que garantiza la estabilidad del sistema es:
% \begin{align}
% k_{P2} &= 0 \label{eq:kp2_final} \\
% k_{I2} &= 1 \label{eq:ki2_final} \\
% k_{P3} &= 1 \label{eq:kp3_final} \\
% k_{I3} &= 1 \label{eq:ki3_final}
% \end{align}
% A_2 = \left.\frac{\partial f}{\partial z}\right|_{z=\bar{z}} = \begin{bmatrix}
% 0 & -k_{I3} & -k_{P3} & 1 \\
% 0 & 0 & 1 & 0 \\
% 0 & 0 & -\frac{f_r}{J} + {\gamma \over J} & -\frac{3}{2J}p\phi \\
% -\frac{k_{I2}}{L} & \frac{k_{P2}k_{I3}}{L} & \frac{k_{P2}k_{P3}}{L} & -\frac{R+k_{P2}}{L}
% \end{bmatrix} \label{eq:A2_symbolic}
% \end{equation}

% y usando la regla de la cadena:
% \begin{align}
% \gamma :=&\left. \frac{r\kappa_1}{v_w} \frac{\partial }{\partial \lambda}\left[\frac{C_{p,0}}{\lambda }\right]\right|_{\omega_m=\omega^*_m,\lambda=\lambda^\star}\nonumber\\
% =&\left. \frac{r\kappa_1}{v_w}\frac{\partial\lambda}{\partial \omega_m}\left[{1\over \lambda}\frac{\partial C_{p,0}}{\partial\lambda\nonumber }-\frac{1}{\lambda^2}C_{p,0}\right]\right|_{\omega_m=\omega^*_m,\lambda=\lambda^\star}\\
% =&  \frac{r^2\kappa_1}{v_w^2}{1\over \lambda^\star}\left[\left.\frac{\partial C_{p,0}}{\partial \lambda}\right|_{\lambda=\lambda^\star}-{1\over \lambda^\star} C_{p,0}(\lambda^\star) \right]\nonumber \\
% =& - \frac{r^2\kappa_1}{v_w^2}{1\over (\lambda^\star)^2}  C_{p,0}(\lambda^\star)
% \label{eq:gamma}
% \end{align}

% Puesto que los parámetros son positivos y $C_{p,0}>0$, entonces $\gamma<0$. A partir de lo anterior, 
% es posible enunciar el siguiente lema. Asismismo, es posible conlcuir que la entrada (3,3) de la 
% matriz $A_2$ toma valores negativos, esto es, $A_{2}^{(3,3)}<0$.\\

% \begin{lemma}
%     Considere el sistema $\Sigma_2^{\mathrm{lin}}$ en \eqref{dz} que consiste en la linealización 
%     de $\Sigma_2$. El punto de equilibrio $z=\bar z$, con la constante $\bar z$  definida en 
%     \eqref{barz}, es exponencialmente estable si y solo si .... 

%     \begin{proof}
%         La estabilidad del sistema  \eqref{dz} se puede inferir de la estabilidad de la matriz%\footnote{\textcolor{blue}{Note que la matriz $\bar A_2$ se puede obtener a partir de un cambio de variables de \eqref{barz}. Por ser muy sencillo, el procedimiento se omitirá.}}
%         $$\bar A_2:=\begin{bmatrix}
%             0 & -k_{I3} & -k_{P3} & 1 \\
%             0 & 0 & 1 & 0 \\
%             0 & 0 & -\bar \gamma_1  & -\bar\gamma_2 \\
%             -K_{I2}' & k_{P2}'k_{I3} & k_{P2}'\textcolor{red}{\underbrace{K_{P3}'}_{k_{\text P3} \text ?}}& -(\frac{R}{L}+k_{P2}')
%         \end{bmatrix}$$
%         donde $\bar\gamma_1:=\frac{f_r-\gamma}{J}>0,$ $\bar \gamma_2:=\frac{3}{2J}p\phi>0$, $K_{I2}'=\frac{k_{I2}}{L}$, $K_{P2}'=\frac{K_{P2}}{L}$,  .
%         \textcolor{red}{Comentario: $k_P3$ va sin cambio de variable, cierto?}
%     \end{proof}
% \end{lemma}

% El polinomio característico de la matriz $\bar{A_2}$ es 

% \begin{equation}\label{eq:poly_sigma_2}
%     \begin{split}
%         p_2(s) =& s^4 + \left(\frac{R}{L} + \bar{\gamma_1} + k_{\textrm P2}'\right)s^3 + \ldots 
%         \\ 
%         &\ldots + \left( \frac{R\bar{\gamma_1}}{L} + k_{\textrm I2}' + \bar{\gamma_1}k_{\textrm P2}' + \bar{\gamma_2}k_{\textrm P2}'k_{\textrm P3} \right)s^2 + \ldots 
%         \\
%         &\ldots + \left( \bar{\gamma_1}k_{\textrm I2} + \bar{\gamma_2}k_{\textrm I2}k_{\textrm P3} + \bar{\gamma_2}k_{\textrm I3}k_{\textrm P2} \right)s + \ldots
%         \\
%         &\ldots + \left( \bar{\gamma_2}k_{\textrm I2}k_{\textrm I3} \right)
%     \end{split}
% \end{equation}

% El polinomio característico tiene la forma :
% \begin{equation}\label{eq:poly_sigma_2_form}
%     p_2(s) = a_4s^4 + a_3s^3 +a_2s^2 a_1s + a_0
% \end{equation}

% La tabla de Routh-Hurwitz para un polinomio de cuarto orden requiere que la primera columna tenga 
% todos sus elementos positivos. La tabla tiene la siguiente estructura:

% \begin{equation}
% \begin{array}{c|ccc}
% s^4 & a_4 & a_2 & a_0 \\
% s^3 & a_3 & a_1 & 0 \\
% s^2 & b_1 & b_2 & 0 \\
% s^1 & c_1 & 0 & 0 \\
% s^0 & d_1 & 0 & 0
% \end{array} \label{eq:rh_table}
% \end{equation}
% % \textcolor{blue}{Comentario: Puedes repetir el análisis de RH con la matriz $\bar A_2$? Las k's, R, fr y $\gamma$ son constantes. No hay por que entrar en detalle en cómo está definida la funcion Cp para realizar este análisis. }
% % \textcolor{blue}{
% % Es decir, realizar lo que tienes en (1.65)-(1.69) pero escribiendo de forma simbóilica los parámetros. En el  polinomio caracteristicco deben aparecer los parámetros anteriores.}


% % \hfill{$\square$}

% % ---------------\\
% % \textcolor{blue}{Comentario: No es necesario definir los valores de los parámetros. Eso se puede hacer después. Tal vez en la parte de simulación.}\\
% % -----------------

% % -------------------\\

% La matriz $A_2$ con los valores numéricos es:

% \begin{equation}
% \resizebox{\textwidth}{!}{$
% A_2 = 
% \begin{bmatrix}
% 0 & -k_{I3} & -k_{P3} & 1 \\
% 0 & 0 & 1 & 0 \\
% 0 & 0 & -3.86\times10^{-5} - 0.0015 & -1.5328 \\
% -281.69k_{I2} & 281.69k_{P2}k_{I3} & 281.69k_{P2}k_{P3} & -281.69k_{P2} - 103.55
% \end{bmatrix}
% $}
% \label{eq:A2_numeric_final}
% \end{equation}

% Y su polinomio característico:
% \begin{multline}
% p_2(s) = s^4 + (281.69k_{P2} + 103.55)s^3 + \\ 
% (281.69k_{I2} + 431.76k_{P2}k_{P3} - 0.0015k_{P2} - 0.0004)s^2 + \\
% (431.76k_{I2}k_{P3} - 0.0015k_{I2} + 431.76k_{P2}k_{I3})s + 431.76k_{I2}k_{I3} \label{eq:char_poly}
% \end{multline}

% El polinomio característico tiene la forma:
% \begin{equation}
% p_2(s) = a_3s^4 + a_2s^3 + a_1s^2 + a_0s + a_4 \label{eq:poly_rh}
% \end{equation}

% donde:
% \begin{align}
% a_3 &= 1 \label{eq:a3} \\
% a_2 &= 281.69k_{P2} + 103.55 \label{eq:a2} \\
% a_1 &= 281.69k_{I2} + 431.76k_{P2}k_{P3} - 0.0015k_{P2} - 0.0004 \label{eq:a1} \\
% a_0 &= 431.76k_{I2}k_{P3} - 0.0015k_{I2} + 431.76k_{P2}k_{I3} \label{eq:a0} \\
% a_4 &= 431.76k_{I2}k_{I3} \label{eq:a4}
% \end{align}


% donde:
% \begin{align}
% b_1 &= \frac{a_2a_1 - a_0}{a_2} \label{eq:b1} \\
% b_2 &= \frac{a_2a_4 - 0}{a_2} = a_4 \label{eq:b2} \\
% c_1 &= \frac{b_1a_0 - a_4}{b_1} \label{eq:c1} \\
% d_1 &= a_4 \label{eq:d1}
% \end{align}

% Para garantizar estabilidad, necesitamos:

% \begin{align}
% a_2 &> 0 \label{eq:cond1} \\
% a_1 &> 0 \label{eq:cond2} \\
% a_0 &> 0 \label{eq:cond3} \\
% a_4 &> 0 \label{eq:cond4} \\
% b_1 &> 0 \label{eq:cond5} \\
% c_1 &> 0 \label{eq:cond6}
% \end{align}

% De \eqref{eq:cond1}:
% \begin{equation}
% 281.69k_{P2} + 103.55 > 0 \implies k_{P2} > -0.3676 \label{eq:kp2_cond}
% \end{equation}

% De \eqref{eq:cond4}:
% \begin{equation}
% 431.76k_{I2}k_{I3} > 0 \implies k_{I2}k_{I3} > 0 \label{eq:ki_cond}
% \end{equation}

% Para simplificar el análisis, podemos hacer:
% \begin{equation}
% k_{P2} = 0, \quad k_{I2} = 1, \quad k_{I3} = 1 \label{eq:k_values}
% \end{equation}

% Sustituyendo estos valores en \eqref{eq:cond2} y \eqref{eq:cond3}:
% \begin{align}
% 281.69 + 431.76k_{P3} - 0.0004 &> 0 \label{eq:kp3_cond1} \\
% 431.76k_{P3} - 0.0015 + 0 &> 0 \label{eq:kp3_cond2}
% \end{align}

% De donde obtenemos:
% \begin{equation}
% k_{P3} > 3.48\times10^{-6} \label{eq:kp3_final_cond}
% \end{equation}

% Por lo tanto, un conjunto de valores que garantiza la estabilidad del sistema es:
% \begin{align}
% k_{P2} &= 0 \label{eq:kp2_final} \\
% k_{I2} &= 1 \label{eq:ki2_final} \\
% k_{P3} &= 1 \label{eq:kp3_final} \\
% k_{I3} &= 1 \label{eq:ki3_final}
% \end{align}


%\chapter{Implementación} \label{intro_chapter}

En este capítulo se describe el modelo considerado óptimo y que cumple con los estándares necesarios para llevar a cabo la buena solución del problema. 

\section{Modelo óptimo}
Lorem ipsum dolor sit amet, consectetur adipiscing elit, sed do eiusmod tempor incididunt ut labore et dolore magna aliqua. Ut enim ad minim veniam, quis nostrud exercitation ullamco laboris nisi ut aliquip ex ea commodo consequat.Lorem ipsum dolor sit amet, consectetur adipiscing elit, sed do eiusmod tempor incididunt ut labore et dolore magna aliqua. Ut enim ad minim veniam, quis nostrud exercitation ullamco laboris nisi ut aliquip ex ea commodo consequat.
%\chapter{Pruebas y resultados}

En este capítulo se describen los resultados obtenidos en cada parte de la implementación de la solución así como las pruebas realizadas y los resultados de éstas con el fin de proporcionar la información adecuada para saber si se cumplieron los requerimientos de esta tesina. 
\section{Pruebas realizadas y sus resultados}
Lorem ipsum dolor sit amet, consectetur adipiscing elit, sed do eiusmod tempor incididunt ut labore et dolore magna aliqua. Ut enim ad minim veniam, quis nostrud exercitation ullamco laboris nisi ut aliquip ex ea commodo consequat.
Lorem ipsum dolor sit amet, consectetur adipiscing elit, sed do eiusmod tempor incididunt ut labore et dolore magna aliqua. Ut enim ad minim veniam, quis nostrud exercitation ullamco laboris nisi ut aliquip ex ea commodo consequat.
Lorem ipsum dolor sit amet, consectetur adipiscing elit, sed do eiusmod tempor incididunt ut labore et dolore magna aliqua. Ut enim ad minim veniam, quis nostrud exercitation ullamco laboris nisi ut aliquip ex ea commodo consequat.Lorem ipsum dolor sit amet, consectetur adipiscing elit, sed do eiusmod tempor incididunt ut labore et dolore magna aliqua. Ut enim ad minim veniam, quis nostrud exercitation ullamco laboris nisi ut aliquip ex ea commodo consequat.
%\chapter{Conclusiones} \label{intro_chapter}

En este capítulo se analizan los resultados de las pruebas realizadas para determinar si la solución proporcionada de esta tesis cumple con los requerimientos de la misma. 

\section{Conclusiones pruebas de velocidad y posición}
Lorem ipsum dolor sit amet, consectetur adipiscing elit, sed do eiusmod tempor incididunt ut labore et dolore magna aliqua. Ut enim ad minim veniam, quis nostrud exercitation ullamco laboris nisi ut aliquip ex ea commodo consequat.Lorem ipsum dolor sit amet, consectetur adipiscing elit, sed do eiusmod tempor incididunt ut labore et dolore magna aliqua. Ut enim ad minim veniam, quis nostrud exercitation ullamco laboris nisi ut aliquip ex ea commodo consequat.Lorem ipsum dolor sit amet, consectetur adipiscing elit, sed do eiusmod tempor incididunt ut labore et dolore magna aliqua. Ut enim ad minim veniam, quis nostrud exercitation ullamco laboris nisi ut aliquip ex ea commodo consequat.Lorem ipsum dolor sit amet, consectetur adipiscing elit, sed do eiusmod tempor incididunt ut labore et dolore magna aliqua. Ut enim ad minim veniam, quis nostrud exercitation ullamco laboris nisi ut aliquip ex ea commodo consequat.Lorem ipsum dolor sit amet, consectetur adipiscing elit, sed do eiusmod tempor incididunt ut labore et dolore magna aliqua. Ut enim ad minim veniam, quis nostrud exercitation ullamco laboris nisi ut aliquip ex ea commodo consequat.Lorem ipsum dolor sit amet, consectetur adipiscing elit, sed do eiusmod tempor incididunt ut labore et dolore magna aliqua. Ut enim ad minim veniam, quis nostrud exercitation ullamco laboris nisi ut aliquip ex ea commodo consequat.

%----------------------------------------------------------------------------------------
%	Apéndices
%----------------------------------------------------------------------------------------

\appendix
%\input{Apendices/Apendices}


%----------------------------------------------------------------------------------------
%	Bibliografía
%----------------------------------------------------------------------------------------
%\printbibliography[notcategory=cited, title={Referencias}]
%\printbibliography[title = {Referencias}]
\renewcommand\bibname{Referencias}
\bibliography{Referencias/Documentos/references}
\bibliographystyle{ieeetr}

\end{document}